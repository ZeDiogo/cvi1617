\documentclass{article}
\usepackage[left=3cm,right=3cm,top=0cm,bottom=2cm]{geometry} % page settings
\usepackage{amsmath} % provides many mathematical environments & tools
\usepackage[pdftex]{graphicx}
\usepackage{fontenc}
\usepackage{selinput}

\setlength{\parindent}{0mm}

\begin{document}

\title{CVI: Gr\'aficos Segundo Trabalho}
\author{
\textsc{Jos\'e Oliveira}
\normalsize 75255 \\
\textsc{Andr\'e Miguel}
\normalsize 76445 \\
\textsc{Jo\~ao Costa}
\normalsize 76476 \\
}
\date{\today}
\maketitle

\subsection*{Breve Introdu\c{c}\~ao}
Neste breve documento s\~ao apresentados os gr\'aficos gerados em tempo de execu\c{c}\~ao no MATLAB. Tal como se pode verificar, os gr\'aficos terminam na frame 2650 uma vez que os computadores em que execut\'amos o programa tinham pouca mem\'oria impedindo a total execu\c{c}ão do nosso trabalho.
Especificamente no gr\'afico do "recall", o valor constante de 1 deve-se ao facto de nunca termos falsos negativos. Como \'e' estimado o valor seguinte atrav\'es do "time validation", nunca contamos falsos negativos. Segundo a f\'ormula em baixo, com falsos negativos (FN) sempre 0, o valor da fun\c{c}\~ao ser\'a sempre 1.
\begin{align*}
    recall &= \frac{TP}{TP + FN}
\end{align*}

\subsection*{Gr\'aficos Grupo 7}

\begin{figure}[ht!]
\centering
\includegraphics[width=100mm]{graficos2.png}
\caption{Precision \label{overflow}}
\end{figure}

\begin{figure}[ht!]
\centering
\includegraphics[width=100mm]{graficos3.png}
\caption{Recall \label{overflow}}
\end{figure}

\begin{figure}[ht!]
\centering
\includegraphics[width=100mm]{graficos1.png}
\caption{IoU \label{overflow}}
\end{figure}

\end{document}